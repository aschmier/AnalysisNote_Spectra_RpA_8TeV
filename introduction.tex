\section{Introduction}
\label{chap:Introduction}

At the CERN LHC, protons and ions are collided at relativistic speeds in order to study the behavior and processes of quantum chromodynamics (QCD). One such process is the production of collimated sprays of particles called jets resulting from the hard scattering of two partons. Jets are an important probe that provide a window into the early stages of the collision.

Measurements in small systems such as pp and p--Pb collisions are important in order to provide constraints on nuclear PDFs and the strong coupling constant $\alpha_{S}$ \cite{CMSPDFConstraints}. Measurements at different center of mass energies can improve constraints on these objects and find deviations from their predictions. In addition, jet production in proton-proton collisions can be used as a reference for which to compare more complex systems, such as the environments produced in p-Pb and Pb-Pb collisions, where cold nuclear matter effects and a strongly-interacting medium are believed to play a role. While such a medium is not expected in small collision systems, recent studies still suggest the presence of collectivity.