\documentclass{cernphprep}
\usepackage{xcolor,pict2e,curve2e}
\begin{document}
\begin{titlepage}
\PHnumber{2005--49}
\PHdate{30 February 2005}
\title{Test-beam performance of a tracking TRD prototype}

\begin{abstract}
Each paper should be preceded by a short abstract of not more 
than 150~words, which should be written as a single paragraph 
and should not contain references.
\end{abstract}

\vfill

\begin{Authlist}
J.T.~Shank, J.S.~Whitaker
\Instfoot{a1}{Boston University, Boston, Mass., USA.}
V.A.~Polychronakos,  V.~Radeka,  D.~Stephani
\Instfoot{a2}{Brookhaven National Laboratory, Upton, NY, USA.}
H.~Beker,  R.K.~Bock,  M.~Botlo,  L.~DiCiaccio\Aref{c}, 
M.~Koratzinos,  A.~Lopez\Aref{*}
\Instfoot{a3}{CERN, Geneva, Switzerland.}
W.J.~Willis
\Instfoot{a4}{Columbia University, New York, NY, USA.}
T.~Akesson
\Instfoot{a5}{Lund University, Lund, Sweden.}
V.~Chernyatin,  B.~Dolgoshein,  P.~Nevski
\Instfoot{a6}{Moscow Physical Engineering Institute, Moscow, USSR.}
\end{Authlist}
\Collaboration{Our collaboration}
\vfill
\Submitted{(Submitted to some important Journal)}
\Conference{To be presented at the conference in ...}
\Note{NOTE: not yet final}
\Dedication{Dedicated to AAA}
\Anotfoot{c}{Univ. di Roma II, `Tor Vergata', Rome, Italy.}
\Anotfoot{*}{On leave from Fac. de Ciencias F\'\i sicas,
Univ. Complutense, Madrid, Spain.}
\ShortAuthor{J.T.~Shank \emph{et al.}}
\end{titlepage}

\section{Introduction}

The biological effect of radiation depends on the quality of 
the radiation as well as on the amount of energy absorbed. Evidence 
suggests that this quality dependence is primarily caused by 
the differences in rates of energy loss~\cite{Raby1966,Dupont1961}. 
The general criteria that have been used are:

\begin{enumerate}
\item[i)]
      to terminate the iteration when the residue between iterated 
      and experiment values is of the order of experimental 
      errors~\cite{Raby1966,Appleman1959,vanBerg1965,Bryant1985,Allen1977};
\item[ii)]
      to terminate when the smoothest solution has been obtained.
\end{enumerate}

For an overall appreciation of the work carried out in this field, 
see Refs.~\cite{Keil1969} and \cite{Guignard1983}.

\section{Examples of figures and tables}

\subsection{Including tabular material}

Smith tabulated the average values of mean linear energy transfer
(LET) obtained by the different methods used in
Ref.~\cite{Appleman1959}, and these are reproduced in
Table~\ref{tab:LET}. Note that a table is produced with \LaTeX's
\texttt{table} environment, and that the caption should be positioned
\emph{above} the tabular material.

\begin{table}[!h]
\begin{center}
\caption{Calculated mean LET values in water (keV/mm)}
\label{tab:LET}
\begin{tabular}{|p{6cm}|c|c|}
\hline
Radiation                    & Smith$^{a)}$    & Jones\\
\hline
1 MeV \ensuremath{\alpha}    &                  &      \\
200 kVp X-rays total         & 3.25             & 1.79 \\
\hline
200 kVp X-rays (primary)     & 2.60             & 1.48 \\
\hline
\multicolumn{3}{@{}l}{$^{a)}$ \footnotesize J. Smith, Rad. Res. 1 (1956) 234}
\end{tabular}
\end{center}
\end{table}

Table~\ref{tab:LET} is reproduced from the publication mentioned
earlier and shows the good agreement between predictions and
calculations.  Comparison should be made with the decay curves shown
in Figs.~6 and 7 of Ref.~\cite{vanBerg1965}, and further information
is given in Section~\ref{sec:curvature} and Appendix~\ref{sec:app}.

\subsection{Including figures}

Figures are defined with the \texttt{figure} environment. The source
for producing the figure can be included inline, as in
Fig.~\ref{fig:sincos} which shows how the $\sin$ and $\cos$ functions
evolve with the help of the \texttt{picture} environment and the
\texttt{pict2e} and \texttt{curve2e} packages.

\begin{figure}[ht]
\begin{center}
\setlength\unitlength{1cm}
\begin{picture}(6.6,3)(0,-1.5)
  \put(0,-1.5){\vector(0,3){3}}
  \put(0,0){\Vector(6.6,0)}
  \put(6.3,0.1){$\theta$}
  \put(0.05,-0.3){$0$}% 0
  \put(2.9,-0.3){$\pi$}% pi
  \put(6.23,-0.3){$2\pi$}% 2 pi
  \color{red}\put(0.1,1.1){$\cos\theta$}
  \Curve(0,1)<1,0>(1.570796,0)<1,-1>%
  (3.1415924,-1)<1,0>(6.283185,1)<1,0>%
  \color{blue}\put(1.65,1.1){$\sin\theta$}
  \Curve(0,0)<1,1>(1.570796,1)<1,0>%
  (4.712389,-1)<1,0>(6.283185,0)<1,1>%
\end{picture}
\end{center}
\caption{The $\sin$ and $\cos$ functions}
\label{fig:sincos}
\end{figure}

Figures can also be imported, in EPS or PDF, PNG and JPEG format,
depending on whether you run \texttt{latex} or
\texttt{pdflatex}. Figure~\ref{fig:extpict} show how to include a
picture with the \verb|\includegraphics| command of \LaTeX's
\texttt{graphicx} package.

\begin{figure}[ht]
\centering
\includegraphics[width=.5\linewidth]{picture}
\caption{Including a figure from an external file}
\label{fig:extpict}
\end{figure}

\subsection{Examples of equations}
\label{sec:curvature}

Equation~\ref{eq:a1} representing a straight line at an angle 
$\theta$, is 
\begin{equation}
n^k(h)u=\lambda h t g q_k \label{eq:a1}  
\end{equation}
and
\begin{equation}
n^k(h) =k h \frac{k}{32}  \label{eq:a2}  
\end{equation}
where:
\begin{itemize}
\item $\lambda$ is the distance between two consecutive sweep lines,
\item $u$ is the least count.
\end{itemize}

We consider a parabola, the tangent of which is parallel to the 
vertical axis of the main matrix. A sufficiently good approximation 
to a parabola [see Eq.~\ref{eq:a3}] drawn inside the matrix can be given 
by the following formula:

\begin{equation}
n_q = \alpha q^2 \quad \alpha=\frac{\lambda^2}{3 Ru} \label{eq:a3}  
\end{equation}
where $R$ is the radius of curvature. We have the following 
relationship:
\begin{equation}
n_q + \mu_q \text{ with } \mu_q = \alpha (2q + 1).   \label{eq:a4}  
\end{equation}

\section{Conclusion}

The theoretical considerations presented have been confirmed 
by their close agreement with the results of practical experiments. 
An account of the earlier work carried out in this field can 
be found in the bibliography.

It is expected that in the next few years many new results will 
be published, since a significant number of new experiments have 
recently been launched.

\section*{Acknowledgements}

We wish to thank C. Brown for his most enlightening comments 
on this topic.

\begin{thebibliography}{99}
\bibitem{Raby1966}
  J.M. Raby, Biophysical aspects of radiation quality, International 
  Atomic EnergyAgency, Technical Reports Series No. 58 (1966).
\bibitem{Dupont1961}
  J.-P. Dupont, Proc. Int. Conf. on Radiation Hazards, Columbia, 
  1960 (Academic Press Inc., New York, 1961), Vol. II, p. 396.
\bibitem{Appleman1959}
  H. Appleman et al., J. Med. Biol. \textbf{8} (1959) 911.
\bibitem{vanBerg1965}
  E. van Berg, D. Johnson and J. Smith, Rad. Res. \textbf{5} (1965) 
  215.
\bibitem{Bryant1985}
  P. Bryant and S. Newman (eds.), The generation of high fields, 
  CAS--ECFA--INFN Workshop, Frascati, 1984. (ECFA 85/91, CERN 85/07, 
  CERN, Geneva, 1985).
\bibitem{Allen1977}
  M.A. Allen et al., IEEE Trans. Nucl. Sci. \textbf{NS--24} (1977) 
  1780.
\bibitem{Keil1969}
  E. Keil, Nucl. Instrum. Methods \textbf{100} (1969) 419.
\bibitem{Guignard1983}
  G. Guignard, CERN LEP--TH/83--38 (1983).
\end{thebibliography}

\section*{Bibliography}

I.C. Percival and D. Richards, Introduction to dynamics (Cambridge 
University Press, 1982).\newline
L. Garrido (ed.), Dynamical systems and chaos: Proc. Sitges 1982, 
Lecture Notes in Physics No.~179 (Springer, Berlin, 1983).

\appendix
\section{Construction on a flat site}
\label{sec:app}

\subsection{General considerations}

Following on ECFA recommendation, the project described in this 
report is based on the assumption that the machine is built close 
to the present CERN site, and has been taken into account in 
the cost estimate.

\subsection{Effects on the construction}

The general layout of the machine would be very similar to that 
shown in the main body of the report.
\end{document}
